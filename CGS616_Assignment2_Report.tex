\documentclass[11pt, a4paper, twocolumn]{article}

% ─── Packages ───
\usepackage[margin=0.65in, top=0.7in, bottom=0.7in, columnsep=0.2in]{geometry}
\usepackage{amsmath, amssymb, amsfonts}
\usepackage{graphicx}
\usepackage{booktabs}
\usepackage{hyperref}
\usepackage{float}
\usepackage{enumitem}
\usepackage{caption}
\usepackage{subcaption}
\usepackage[numbers]{natbib}
\usepackage{fancyhdr}
\usepackage{titlesec}
\usepackage{xcolor}
\usepackage{tcolorbox}
\usepackage{microtype}

\captionsetup{font=scriptsize, skip=2pt}
\setlength{\abovecaptionskip}{3pt}
\setlength{\belowcaptionskip}{0pt}

\hypersetup{colorlinks=true, linkcolor=blue!70!black, citecolor=blue!70!black, urlcolor=blue!70!black}

\pagestyle{fancy}
\fancyhf{}
\rhead{\scriptsize CGS616 --- Assignment 2}
\lhead{\scriptsize Dynamic Computational Model}
\cfoot{\scriptsize\thepage}
\renewcommand{\headrulewidth}{0.3pt}

\titleformat{\section}{\normalsize\bfseries}{\thesection.}{0.4em}{}
\titleformat{\subsection}{\small\bfseries}{\thesubsection}{0.4em}{}
\titlespacing{\section}{0pt}{6pt}{2pt}
\titlespacing{\subsection}{0pt}{4pt}{1pt}

\setlist[itemize]{nosep, leftmargin=*, topsep=1pt, itemsep=0pt, parsep=0pt}
\setlist[enumerate]{nosep, leftmargin=*, topsep=1pt, itemsep=0pt, parsep=0pt}

\tcbset{fonttitle=\scriptsize\bfseries, fontupper=\scriptsize, boxsep=2pt, left=2pt, right=2pt, top=1pt, bottom=1pt}

% ─── Document ───
\begin{document}

% ─── Title (single column) ───
\twocolumn[{%
\begin{center}
{\large\bfseries CGS616 Assignment 2: Dynamic Computational Model of Latent Cognitive States}\\[2pt]
{\small Markov Chain / HMM-Based Behavioral Modeling with Reinforcement Learning}\\[3pt]
{\small Riya Sanket Kashive, 220902}
\end{center}
\vspace{4pt}
\hrule
\vspace{6pt}
}]

% ══════════════════════════════════════════════════════════════════════
\section{Introduction}
% ══════════════════════════════════════════════════════════════════════

This assignment presents a dynamic computational model explaining how \textbf{latent cognitive states} evolve over time and generate \textbf{observable behaviors} during crisis events. Building on Assignment~1's ABC framework applied to crisis-related social media data, we construct a probabilistic state transition model using Hidden Markov Models (HMM) with sigmoid-based transitions and a Q-Learning reinforcement learning extension.

\textbf{Dataset:} CrisisNLP Combined Dataset (41,149 tweets; 11 disasters; 10 countries). Tweets carry timestamps enabling temporal analysis.

\textbf{Model Variables:} $S_t$ = latent cognitive state (0 = passive/impact, 1 = active/engaged); $X_t$ = stimulus antecedent (Global North = 1, South = 0); $B_t$ = observable behavior (Action, Impact, Info); $\epsilon_t \sim \mathcal{N}(0,\tau^2)$.

% ══════════════════════════════════════════════════════════════════════
\section{Latent State Model}
% ══════════════════════════════════════════════════════════════════════

We model two conditional distributions:
\vspace{-2pt}
\begin{align}
P(S_t \mid S_{t-1}, X_t) &\quad \text{(State Transition)} \\
P(B_t \mid S_t) &\quad \text{(Emission)}
\end{align}

\textbf{State Mapping:} Action ($B_t{=}0$) and Info ($B_t{=}2$) map to Active state ($S_t{=}1$); Impact ($B_t{=}1$) maps to Passive ($S_t{=}0$). Tweets are sorted by \texttt{tweet\_time\_utc} within each disaster to preserve the Markov property.

\subsection{State Transition \& Emission}

\vspace{-2pt}
\begin{equation}
P(S_{t+1}{=}1 \mid S_t, X_t) = \sigma(\beta X_t - \delta S_t + \epsilon_t), \quad \sigma(z)=\frac{1}{1+e^{-z}}
\end{equation}
\vspace{-4pt}
\begin{equation}
P(B_t{=}k \mid S_t{=}s) = \frac{e^{\phi_{s,k}}}{\sum_{j=0}^{2} e^{\phi_{s,j}}}
\end{equation}

Parameters: $\beta$ = stimulus sensitivity; $\delta$ = decay/recovery rate; $\tau$ = noise s.d. Long-run equilibrium: $S^* = \frac{\beta}{\delta} \cdot X$.

% ══════════════════════════════════════════════════════════════════════
\section{Empirical Transition Analysis}
% ══════════════════════════════════════════════════════════════════════

\begin{table}[H]
\centering\scriptsize
\caption{Empirical transition matrices $P(S_{t+1} \mid S_t)$}
\setlength{\tabcolsep}{4pt}
\begin{tabular}{@{}lcccc@{}}
\toprule
& \multicolumn{2}{c}{\textbf{North} ($n{=}24{,}147$)} & \multicolumn{2}{c}{\textbf{South} ($n{=}17{,}002$)} \\
\cmidrule(lr){2-3}\cmidrule(lr){4-5}
& $\to$Pass & $\to$Act & $\to$Pass & $\to$Act \\
\midrule
Passive & 0.4294 & 0.5706 & 0.3370 & 0.6630 \\
Active  & 0.3286 & 0.6714 & 0.2592 & 0.7408 \\
\bottomrule
\end{tabular}
\end{table}

Both regions tend to remain in or transition to the active state; Global South shows slightly higher active-state persistence.

% ══════════════════════════════════════════════════════════════════════
\section{Parameter Estimation (MLE)}
% ══════════════════════════════════════════════════════════════════════

\textbf{Objective:} Minimize negative log-likelihood:
\vspace{-2pt}
\begin{equation}
-\log\mathcal{L}(\theta) = -\sum_t \log P(B_t|S_t) - \sum_{t>0} \log P(S_t|S_{t-1},X_t)
\end{equation}

Optimized via L-BFGS-B (\texttt{scipy.optimize.minimize}), with $\beta\in[-5,5]$, $\delta\in[0.01,5]$, $\tau\in[0.01,5]$, emission logits $\in[-5,5]$.

\begin{table}[H]
\centering\scriptsize
\caption{MLE-estimated parameters}
\setlength{\tabcolsep}{4pt}
\begin{tabular}{@{}lcc@{}}
\toprule
\textbf{Parameter} & \textbf{Global North} & \textbf{Global South} \\
\midrule
$\beta$ (Sensitivity) & 0.5583 & 0.5000 \\
$\delta$ (Decay Rate)  & 0.0100 & 0.0100 \\
$\tau$ (Noise)         & 1.0152 & 1.0081 \\
\midrule
Log-Likelihood & $-25{,}854.02$ & $-19{,}315.08$ \\
AIC & $51{,}722.04$ & $38{,}644.15$ \\
BIC & $51{,}778.69$ & $38{,}698.34$ \\
\bottomrule
\end{tabular}
\end{table}

\begin{table}[H]
\centering\scriptsize
\caption{Fitted emission probabilities $P(B_t \mid S_t)$}
\setlength{\tabcolsep}{3pt}
\begin{tabular}{@{}lcccccc@{}}
\toprule
& \multicolumn{3}{c}{\textbf{Global North}} & \multicolumn{3}{c}{\textbf{Global South}} \\
\cmidrule(lr){2-4}\cmidrule(lr){5-7}
& Act & Imp & Info & Act & Imp & Info \\
\midrule
Passive ($S{=}0$) & 0.0000 & 0.9933 & 0.0067 & 0.0000 & 0.9933 & 0.0067 \\
Active ($S{=}1$)  & 0.6523 & 0.0023 & 0.3454 & 0.7021 & 0.0020 & 0.2959 \\
\bottomrule
\end{tabular}
\end{table}

\begin{figure}[H]
\centering
\includegraphics[width=\columnwidth]{plot_transition_matrices.png}
\caption{Fitted transition matrix heatmaps}
\end{figure}

% ══════════════════════════════════════════════════════════════════════
\section{Q-Learning RL Extension}
% ══════════════════════════════════════════════════════════════════════

\textbf{Update rule:} $Q_{t+1} = Q_t + \alpha(r_t - Q_t)$, where rewards are: Action $\to +1$, Impact $\to -1$, Info $\to 0$.

\textbf{Softmax behavior:} $P(B_t{=}k) = e^{Q_{t,k}/\tau} / \sum_j e^{Q_{t,j}/\tau}$.

Grid search over $\alpha\in\{0.01,0.05,0.1,0.2,0.3,0.5\}$ and $\tau\in\{0.1,0.5,1.0,2.0,5.0\}$.

\begin{table}[H]
\centering\scriptsize
\caption{Q-Learning results}
\setlength{\tabcolsep}{4pt}
\begin{tabular}{@{}lcc@{}}
\toprule
\textbf{Parameter} & \textbf{Global North} & \textbf{Global South} \\
\midrule
$\alpha$ (Learning Rate) & 0.5000 & 0.0100 \\
$\tau$ (Temperature)     & 5.0000 & 2.0000 \\
\midrule
Log-Likelihood & $-26{,}616.67$ & $-18{,}109.33$ \\
AIC & $53{,}237.34$ & $36{,}222.67$ \\
BIC & $53{,}253.52$ & $36{,}238.15$ \\
\bottomrule
\end{tabular}
\end{table}

\begin{figure}[H]
\centering
\includegraphics[width=\columnwidth]{plot_qvalue_trajectories.png}
\caption{Q-value trajectories over time}
\end{figure}

% ══════════════════════════════════════════════════════════════════════
\section{Model Evaluation}
% ══════════════════════════════════════════════════════════════════════

\begin{table}[H]
\centering\scriptsize
\caption{AIC/BIC comparison (lower is better)}
\setlength{\tabcolsep}{3pt}
\begin{tabular}{@{}lcccc@{}}
\toprule
& \multicolumn{2}{c}{\textbf{AIC}} & \multicolumn{2}{c}{\textbf{BIC}} \\
\cmidrule(lr){2-3}\cmidrule(lr){4-5}
\textbf{Model} & North & South & North & South \\
\midrule
HMM       & 51,722 & 38,644 & 51,779 & 38,698 \\
Q-Learning & 53,237 & 36,223 & 53,254 & 36,238 \\
\bottomrule
\end{tabular}
\end{table}

HMM achieves better fit for Global North; Q-Learning performs marginally better for Global South.

\textbf{5-Fold Cross-Validation:}

\begin{table}[H]
\centering\scriptsize
\caption{CV accuracy across folds}
\setlength{\tabcolsep}{3pt}
\begin{tabular}{@{}lcccccc@{}}
\toprule
& F1 & F2 & F3 & F4 & F5 & \textbf{Mean$\pm$Std} \\
\midrule
North & .763 & .790 & .782 & .776 & .785 & $\mathbf{0.779\pm0.009}$ \\
South & .777 & .789 & .789 & .784 & .790 & $\mathbf{0.786\pm0.005}$ \\
\bottomrule
\end{tabular}
\end{table}

Both regions achieve $\approx$78\% accuracy with low variance across folds, confirming stable generalization.

\begin{figure}[H]
\centering
\includegraphics[width=\columnwidth]{plot_model_evaluation.png}
\caption{Model evaluation: log-likelihood, AIC, and CV accuracy}
\end{figure}

% ══════════════════════════════════════════════════════════════════════
\section{Behavioral Trajectory Simulation}
% ══════════════════════════════════════════════════════════════════════

50 Monte Carlo trajectories of 200 steps each. Initialize $S_0\sim\text{Bern}(0.5)$; at each step emit $B_t\sim P(B_t|S_t)$, then transition $S_{t+1}\sim\text{Bern}(\sigma(\beta X_t - \delta S_t + \epsilon_t))$.
\begin{table}[H]
\centering\scriptsize
\caption{Simulated vs.\ real behavior distributions}
\setlength{\tabcolsep}{3pt}
\begin{tabular}{@{}lcccc@{}}
\toprule
& \multicolumn{2}{c}{\textbf{North}} & \multicolumn{2}{c}{\textbf{South}} \\
\cmidrule(lr){2-3}\cmidrule(lr){4-5}
\textbf{Behavior} & Sim & Real & Sim & Real \\
\midrule
Action & 40.5\% & 41.4\% & 34.6\% & 50.5\% \\
Impact & 38.0\% & 36.5\% & 50.8\% & 28.1\% \\
Info   & 21.6\% & 22.1\% & 14.7\% & 21.4\% \\
\bottomrule
\end{tabular}
\end{table}

Global North simulations closely match real data. Global South discrepancy suggests the binary antecedent does not fully capture developing-region crisis dynamics.

\begin{figure}[H]
\centering
\includegraphics[width=\columnwidth]{plot_real_vs_simulated.png}
\caption{Real vs.\ simulated behavior distributions}
\end{figure}

% ══════════════════════════════════════════════════════════════════════
\section{Psychological Interpretation}
% ══════════════════════════════════════════════════════════════════════

\textbf{$\beta$ (Stimulus Sensitivity):} Both regions show moderate sensitivity ($\approx0.5$); North's slightly higher value suggests marginally greater responsiveness, possibly due to better communication infrastructure.

\textbf{$\delta$ (Decay/Recovery Rate):} Both regions: $\delta = 0.01$ (very low). Once activated, cognitive states are \textbf{highly persistent} --- consistent with the sustained attentional demands of natural disasters.

\textbf{$\tau$ (Noise):} $\approx1.0$ for both regions, reflecting moderate stochastic variability in behavioral responses.

\textbf{$\alpha$ (RL Learning Rate):} Most striking regional difference. North ($\alpha=0.5$): \textbf{rapid habit formation}, driven by fast feedback loops (real-time media). South ($\alpha=0.01$): \textbf{gradual reinforcement}, slower behavioral updating.

\textbf{Softmax Temperature $\tau_\text{RL}$:} Both regions exploratory (North: 5.0, South: 2.0), indicating no strong behavioral fixation.

\textbf{Equilibrium Analysis:} For $X_t{=}1$ (North stimulus): $S^*{=}55.83$, strongly driving active engagement. For $X_t{=}0$ (South stimulus): $S^*{=}0$, tending toward passive/impact states. This reflects structural differences in crisis response capacity between regions.

\begin{figure}[H]
\centering
\includegraphics[width=\columnwidth]{plot_parameter_comparison.png}
\caption{Parameter comparison: Global North vs.\ South}
\end{figure}

% ══════════════════════════════════════════════════════════════════════
\section{Conclusion}
% ══════════════════════════════════════════════════════════════════════

A sigmoid-based HMM effectively captures crisis-related cognitive state dynamics. 

Key findings:
\begin{itemize}
    \item Moderate $\beta\approx0.5$ shows economic context measurably influences behavior
    \item $\delta\approx0.01$ confirms highly persistent crisis schemas
    \item Divergent RL learning rates ($\alpha_\text{N}{=}0.5$ vs.\ $\alpha_\text{S}{=}0.01$) reflect infrastructure and feedback loop differences
    \item Equilibrium analysis reveals Global North structurally drives active engagement while Global South trends toward impact-reporting
    \item $\approx$78\% CV accuracy confirms robust generalization
\end{itemize}
\end{document}